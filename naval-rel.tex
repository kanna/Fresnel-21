\subsection{Naval Relevance}
\label{sec:naval}

The capacity to observe the actual conditions that affect the water
column over the coastal ocean regions of interest in such a way that
these observations can effectively feed operational models with
assimilation, play a critical role for the Navy. These models impact
how regular operations including support for Search And Rescue,
training or landing warfighters on beaches require a measure of
predictive capacity that makes modeling based on remote sensing and
data assimilation from multiple sources, critical. \proj reinforces
this capability, advancing it by the use of other \emph{integrative}
activities like the use of autonomous robotic platforms using AI and
ML to quantify and reduce environmental uncertainty in a very dynamic
environment. While our use of \texttt{HOPS} in the proposed experiment
has been used for the operational support of for Navy exercises, our
augmentation with the use of bio-geochemical (BGC) modeling and the
use of autonomy will leverage observations from autonomous vehicles
and augment traditional methods such as ship-based and opportunistic
measurements using low cost sensors. The resulting output of \proj via
reports and publications will also increase our understanding of
regions with dynamic fields associated with deep submarine canyons in
coastal ecosystems.


