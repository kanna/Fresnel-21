\subsection{Physics}

The geographical domain in which \proj will focus the activities
constitutes a key area for the shaping of several main characteristics
of the shelf/slope oceanography of the northwestern Portuguese and
Spanish. The marked changes in the continental margin topography that
characterize this area play an important role in establishing the
nature of the shelf response to wind forcing or the adjustment of the
poleward slope intensified flow. The area shows several particularly
expressive examples of coastal mesoscale features such as the large
upwelling filament of Cabo Carvoeiro which is one of the larger and
more persistent features of the summer upwelling regime in the western
portuguese coast, with an expression similar to the large filaments
off Cape Finisterre (northwestern Spanish coast), Cape da Roca (near
Lisbon) or Cape S.Vicente (in the southwestern tip of Portugal).

The long and narrow Nazare Canyon in addition promotes a rich variety
of coastal mesoscale and submesoscale processes that not only directly
affect the local ecosystems but can also contribute to impact a much
larger coastal ocean area. Since they are linked to the canyon
dynamics these processes develop on a relatively small area of the
coastal ocean which are associated with specific areas of the canyon
topography. In this way an observation strategy that concentrates
several observation systems on those small areas can bring
considerable insight on the nature of the processes. This
concentration is of particular interest for \proj objectives. TO
CONTINUE
 
 