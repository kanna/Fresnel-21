\section{\proj Operations}
 
% Real-time monitoring systems:

\proj will leverage existing real-time monitoring capacities which were
installed and are operated by Instituto Hidrogr\'{a}fico (\inste). These
include two multi-parametric buoys with satellite transmission of hourly
data sets and two coastal tide gauges installed in the ports of \naz and
Peniche. These systems form the \naz Canyon Observatory \texttt{MONICAN}
which is a subset of the global real-time monitoring infrastructure for
the Portuguese Exclusive Economic Zone operated by \inst \kc{(MONIZEE
  infrastructure, see figure X)}. Each multiparametric buoy is equipped
with:

\begin{enumerate}

  \item A meteorological mast providing hourly measurements of wind speed and
    direction, air temperature, atmospheric pressure and relative
    humidity

  \item A wave sensor providing hourly measurements of standard wave parameters
    (wave spectra at the end of the deployment)

  \item A downward looking 300 kHz RDI Workhorse ADCP installed at 7 m
    depth and providing currents with $sim 2$ m resolution to about 90 m
    depth Surface temperature sensor (SBE AADI Subsurface temperature
    sensors)

\end{enumerate}  


The buoys are also equipped with fluorometers but the response of these
sensors rapidly degrades in about 1-2 weeks, after maintenance,
primarily due to biofouling.

Additional systems could also be available for \proje. A separate
project proposal was submitted at the end of 2020 to a funding program
(EEA Grants Portugal) where a HF radar system has been proposed for the
\naz area. The system, a CODAR Seasonde with two antennas operating at
13 MHz, one installed in \naz and the other in Peniche. A previous test
of a similar system was conducted by \inst in September--November 2011
and showed that surface currents measurements were available for the
complete area to about 70 km from the shore with a 1 km resolution
\kc{(figure)}.
 
  
\subsection{Sea operations}
 
\proj will observe the coastal ocean area of interest off of \naz, in
such a way as to improve the initialization/assimilation fields and
parameter definitions to be used in the physical and biogeochemical
models. To achieve this objective \proj will develop a strategy largely
inspired in the Rapid Environmental Assessment to navy operations
\kc{citation?}. Specifically a total of three phases (weeks 1 to 3) are
planned:
 
Phase 1 (precursor survey):
 
A preliminary characterization of the geographical area of interest will
be conducted during this initial period. The precursor phase will be
preceded by the identification of the prevailing features that
characterize the coastal ocean area of interest from remote sensing
imagery and (if available) from surface currents measured by a HF radar
system. The surface data will allow to identify the main features
present in the area (in SST, Chl and turbidity and currents) and to
select a limited number of key locations in which the surface
information will be complemented by water column observations. In case
no surface data is available (because of cloud cover obstructing optical
remote sensing) the location of these points will be dictated by the
characteristics of the topography of the area, by the prevailing forcing
conditions and by previous knowledge about the primary processes.
 
The precursor survey will combine operations onboard a small boat with
operations conducted onboard a \inst hydrographic vessel that will pass
through the area during this period. 
 
During 2--3 days a small team onboard a small-boat will conduct a set of
measurements in the water column (to about 200 m depth) at the key
location point identified earlier using low cost systems \kc{this is
  vague} and other sensors of interest and water samples collected at
selected depths on those key locations. Sampling with vertical nets will
also be done at some of these positions. Part of the water samples
collected will be used to evaluate the nutrient profiles in the water
column. These samples will be analyzed at the \inst Marine Chemistry and
Pollution laboratories and will be available in about 1--2 days.
Depending on period selected for the exercise these measurements would
help to characterize the nutrient intake to upper levels associated
with intensified upwelling at the canyon head or the inputs to the
coastal ocean associated by freshwater inputs along the coast. Combined
with the information available from remote sensing images, this data will
provide the basis for the initialization of the nutrients fields in the
BGC model.
 
Part of the water samples and samples collected with a vertical net will
be used to characterize the phyto- and zoo plankton communities. These
samples will provide the basis for the initialization of the phyto and
zooplankton compartments of the BGC model.
 
At the same time the hydrographic vessel will pass on the area and
during this period will conduct a very limited number of CTD profiles
and collect vessel mounted ADCP measurements. The analysis of water
samples collected by the small boat campaign at a few locations and same
times will then be used for calibration of the CTD fluorometer and
calibrate the ADCP echo intensity data in zooplankton biomass. These
calibrations will then be used during next phase of the survey

\kc{
This data will also be used to calibrate the response of fluorometers to
be used in response
 
This phase will provide key points to be used in the following stages
such as vertical profiles of temperature and salinity and water samples
collected at selected depths to be used in the definitions of d
will be inspired in the strategies used with real and assimilation
fields. In \proj we propose to extend this This phase will provide the
background knowledge to be used in the numerical model initialization
such as and the definition of some of the basic parameters (such as
light attenuation parameters and so one).. Also during this phase we
calibrate the response of the some of the sensors using}
 
 


\begin{table}[H]
  \centering
  \vspace{-0.5cm}
  \begin{tabular}{|p{4cm}|p{4cm}|p{4cm}|p{4cm}|}\hline 
    % \rowcolor{Gray}
    \bfseries  &\bfseries Week 1 &\bfseries Week 2 &\bfseries Week 3 \\
    \hline
    Main Goals& Precursor survey; 
                To collect data that allow to build initialization
                fields and define model parameters. To calibrate sensor response (phytoplankton from
                fluorometry, zooplankton from VMADCP)& Update Survey 1:
                                                       AUV operations; potential additional measurements using small boats
                                                       and low cost measurements & Update Survey 2:
                                                                                   Ship measurements during dedicated survey, AUV operations from ship and small boats\\
    \hline
    Observations at sea&&&\\
    \hline
    AUVs&&&\\
    \hline
    Gliders&&&\\
    \hline
    UAVs&&&\\
    \hline
    ASVs&&&\\
    \hline
    Ship-based observations& Ship crossing the area, deploy Multipametric buoy M2
                             CTDs/Water Sampling at key positions
                             VMADCP collected in area/key points&& CTD/LADCP profiles
                                                                   Rosette
                                                                   Samples
                                                                   for
                                                                   nutrients/phyto/zoo-plankton
                                                                   VMADCP on transit and in station. 
                                                                   Daily transmission of  CTD casts and VMADCP data to \inst\\
    \hline
    Small boat operations&CTD + fluorometry + light profiles at key
                           stations. Water samples at key points for
                           nutrients/phyto+zoo-plankton Vertical net
                           samples for phyto+zoo plankton&&\\
    \hline    
    Laboratory Analysis&Samples analyzed for Nutrients at \inst Labs.
                         Samples analyzed for phyto/zoo-plankton&&Post cruise:
                                                                   Samples
                                                                   analyzed
                                                                   for
                                                                   Nutrients
                                                                   at
                                                                   \inst
                                                                   Labs
                                                                   Samples
                                                                   analyzed
                                                                   for
                                                                   phyto/zoo-plankton\\ 
    \hline
    Remote Sensing&SST, Chl, Turbidity images (Sentinel) used to
                    identify the main features and select key points for
                    observation. 
                    Surface fields combined with observations in key
                    points to build initial 3D fields to be used in the
                    models.&SST, Chl, altimetry data used in assimilation
                             Turbidity images used to track impacts (if
                             any) of local rivers and guide
                             observations.&SST, Chl, altimetry data used in assimilation
                                            Turbidity images used to
                                            track impacts (if any) of
                                            local rivers and guide
                                            observations.\\
    \hline
    Models&Build initialization and initial assimilation fields. 
            Start model runs (end of the week).&&\\
    \hline
  \end{tabular}
  \caption{Tasks and activities in \proje.}
  \label{tab:tasks}
\end{table}
