) THE EXERCISE:
 
Real-time monitoring systems:
Fresnel will profit from already existent real-time monitoring capacities which were installed and are operated by Instituto Hidrográfico (IH). These include two multiparametric buoys with satellite transmission of hourly data sets and two coastal tide gauges installed in the ports of Nazare and Peniche. These systems form the Nazare Canyon Observatory MONICAN which is a subset of the global real-time monitoring infrastructure for the Portuguese Exclusive Economic Zone operated by Instituto Hidrográfico (MONIZEE infrastructure, see figure X). Each multiparametric buoy is equipped with:
 A meteorological mast providing hourly measurements of wind speed and direction, air temperature, atmospheric pressure and relative humidity.
A wave sensor providing hourly measurements of standard wave parameters (wave spectra at the end of the deployment)
A downward looking 300 kHz RDI Workhorse ADCP installed at 7 m depth and providing currents with about 2m resolution to about 90m depth
Surface temperature sensor (AADI
Subsurface temperature sensors (SBE)
 The buoys are also equipped with fluorometers but the response of these sensors is rapidly degraded (in about 1-2 weeks) due to biofouling.

Additional systems could also be available at the moment of the exercise. A project proposal was submitted by the end of 2020 to a funding program (EEA Grants Portugal). In the framework of this proposal a HF radar system is proposed to be installed in the Nazare area. The system will be CODAR Seasonde system with two antennas operating at 13 MHz one installed in Nazare and the second one in Peniche. A previous test of a similar system was conducted by IH in September- November 2011 and showed that surface currents measurements were available for the complete area to about 70km from the shore with a 1 km resolution (figure).
 
 
 
 
 
 
2. Sea operations
 
As described in the Introduction section one of the objectives of \pro
will be to observe the coastal ocean area of interest in such a way to
improve the initialization/assimilation fields and parameter definitions
to be used in the physical and biogeochemical models. To achieve this
objective \pro will develop following a strategy largely inspired in the
Rapid Environmental Assessment to navy operations.  Specifically a total
of three phases (weeks 1 to 3) are
 
Phase 1 (precursor survey):
 
A preliminary characterization of the geographical area of interest will
be conducted during this initial period. The precursor phase will be
preceded by the identification of the prevailing features that
characterize the coastal ocean area of interest from remote sensing
imagery and (if available at that moment) from surface currents measured
by an HF radar system. The surface data will allow to identify the main
features present in the area (in SST, Chl and turbidity and currents)
and to select a limited number of key locations in which the surface
information will be complemented by water column observations. In case
no surface data is available (for example due to cloud coverage) the
location of these points will be dictated by the characteristics of the
topography of the area, by the prevailing forcing conditions and by
previous knowledge about the main processes.
 
The precursor survey will combine operations onboard a small boat (semi-rigid boat or an opportunity boat) with operations conducted onboard IH hydrographic vessel that will pass on the area during this periods.
 
During 2-3 days a small team onboard a semi-rigid boat (or equivalent) will conduct a set of measurements in the water column (to about 200m depth) at the key location point identified earlier using low cost systems and other sensors of interest. Water samples will be collected at selected depth on those key locations. Sampling with vertical nets will also be collected at some of these positions. Part of the water samples collected will be used to evaluate the nutrients profiles in the water column. These sample will be analyzed at IH Marine Chemistry and Pollution laboratories and will be available in about 1-2 days. Depending on period selected for the exercise (September/October 2021, March/April 2022) these measurements would help to characterize the nutrients intake to upper levels associated with intensified upwelling at the canyon head or the inputs to the coastal ocean associated by freshwater inputs along the coast. Combined with the information available from remote sensing images this data will provide the basis for the initialization of the nutrients fields in the BGC model.
 
Part of the water samples and samples collected with a vertical net will be used to characterize the phyto- and zoo plankton communities. These samples will provide the basis for the initialization of the phyto and zooplankton compartments of the BGC model.
 
At the same time the hydrographic vessel will pass on the area and during this period will conduct a very limited number of CTD profiles and collect vessel mounted ADCP measurements. The analysis of water samples collected by the small boat campaign at the small locations and same times will then be used for calibration of the CTD fluorometer and calibrate the ADCP echo intensity data in zooplankton biomass. These calibrations will then be used during next phase of the survey
 
 
This data will also be used to calibrate the response of fluorometers to be used in  response
 
This phase will provide key points to be used in the following stages such as vertical profiles of temperature and salinity and water samples collected in at selected depths to be used in the definitions of  d will be inspired in the strategies used in the  with real and assimilation fields  Fresnel we propose to extend this
This phase will provide the background knowledge to be used in the numerical model initialization such as and the definition of some of the basic parameters (such as light attenuation parameters and so one).. Also during this phase we calibrate the response of the some of the sensors using
 
 

