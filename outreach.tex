\section{Reporting \& Outreach}

\proj will generate a final report to ONR and ONR Global of the
operations during the course of the field experiment. Detailed
analysis and results from the experiment will be published in
peer-reviewed conferences and journals. Targeted Peer-reviewed
journals including \emph{Science Robotics, J. Field Robotics,
  Int. J. of Robotics Research, AI Journal, Autonomous Robots,
  Oceanography, PLOSone, Frontiers} and high-impact conferences such
as \emph{AAAI, IJCAI, ISER, ICAPS, RSS, IROS, ICRA and IEEE AUV} where
we have published frequently, will be the foci of our efforts.

\inst has had continued and extensive engagement over the years with the
local authorities including the \naz city officials and local fishermen.
As a result we will engage the fishing community to carry our low cost
temperature/density sensors and also help provide access to the team to
be able to commute to the islands when necessary, subject to health
protocols. In addition, we expect to engage local high school students
prior and after the experiment, to entice them into thinking about
careers in science and technology, much as we have done in the past with
success in another ONR funded
program~\footnote{\url{https://sunfish.lsts.pt/en/outreach/education}}.

With the world-wide interest in the \naz area, \proj will also work
with the local authorities and \univ outreach department to showcase
our work in Portuguese and English language media outlets. We will
invite ONR and other ocean science sponsors from the US and government
and private philanthropies to observe the field experiment, in-person
if possible. The PI also expects to involve the US Embassy in outreach
activities \footnote{Rajan will be on a US State Dept. Fullbright
  fellowship in Lisbon, in Spring 2022.}. We are also investigating
ways in which portions of the experiment can be livestreamed and made
available via Youtube. Imagery will be uploaded to Twitter, Facebook
and Instagram as is routine at the \ls lab~\footnote{See
  \url{https://lsts.pt}} with appropriate context in both the
Portuguese and English.


