\section{Current/Pending Project and Proposal Submissions}
\label{sec:current}

No proposals related to \proj have been submitted to any US government
agency. However, this proposal is in part to demonstrate a portion of
a much larger concept in the form of the \met project\footnote{See US
  National Academies 'Ocean Shot' presentation at
  \url{https://vimeo.com/510346249}} which is being proposed to
private philanthropies for funding. \met aims to provide a new
paradigm of portable coastal ocean observation using space, aerial,
surface and underwater vehicles, coupled to assimilative models.

% \subsection{Relevant Experience}

% A description of the Offeror's Government contracts (both prime and
% major subcontracts) received during the past three (3) years, which
% are similar in nature to the effort being proposed.  Include contract
% number and the name, phone number and email address of the technical
% point of contact.  If necessary, add additional rows.

The PI and co-PI do not have any current US govt. contracts or have
had any US govt. contracts during the past 3 years. The PI has the
following pending proposal:

\vspace{+0.5cm}
\begin{description}[noitemsep,topsep=0pt,parsep=0pt,partopsep=0pt]

\item [Title:] \textbf{CHAARMS}: Carbon sequestration of Harmful algal
  blooms using Autonomy, Artificial intelligence, Remote sensing and
  ModelS

\item [Summary:] Pelagic \sar is a floating macroalga that is endemic to
  the aptly named Sargasso Sea in the subtropical Atlantic Ocean.
  Massive blooms of this alga have formed regularly in the Tropical
  Atlantic and washed up on the beaches causing substantial economic
  harm from the loss of tourism and fishing. Given the magnitude of this
  excess
  biomass, these blooms of \sar are now considered as Harmful Algal
  Blooms. It is estimated that \sar accounts for about 3 million metric
  tonnes of Carbon in the surface of the Atlantic Ocean of which about 1
  million metric tonnes is due to the new and excess blooms that have
  occurred nearly every year since 2011. The \textbf{C}arbon
  sequestration of \textbf{H}armful algal blooms using
  \textbf{A}utonomy, \textbf{A}rtificial intelligence, \textbf{R}emote
  sensing and \textbf{M}odel\textbf{S} (\projce) project proposes to help
  address one of the most pressing problems facing humanity today --
  that of climate change due to increasing buildup of
  CO\textsubscript{2} in the atmosphere -- by sequestering carbon
  associated with \sar to the deep ocean where it will remain for
  centuries. A monetizable and significant byproduct of this process in
  of itself will be the mitigation of the impact of a Harmful Algal
  Bloom washing up on beaches.

\item [Source:] NSF
\item [Status:] Pending
\item [Amount of Funding:] \$750K -- Phase 1
\item [\%-age of effort:] 25\%  
\item [Identity of prime Offeror:] \orge, LLC
\item [Subawardees:] MIT, Columbia Univ., Georgia Tech, Nova
  Southeastern Univ. of Florida, Univ. of Porto, Univ. of Vigo, Inst. of
  Astrophysics -- Gran Canaria, KTH
\item [Technical contact:] Kanna Rajan
\item [Administrative/business contact:] Linda Holje (lholje@sift.net)
\item [Period of performance:] Sept 2021 -- August 2022
\item [Proposed time on \proje:] 50\%
\item [Relationship of proposed project:] The overall concept of closed
  loop environmental understanding of the upper water-column and the
  tie-in into autonomous systems is related. However, the environmental
  domain and locations are distinct. \proj will study the
  bio-geochemical processes which are not the subject of effort for \projce.
  
\end{description}
