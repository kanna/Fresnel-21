\subsection{Understanding bio-physical Coupling}

% (Understand the dynamics bio-physical coupling of oceanographic processes and biological productivity)

In marine environments driven by physical forcing, where abrupt
topographies (e.g. shelf break, canyons) interact with highly dynamic
oceanographic processes, higher nutrient availability due to enhanced
mixing results in higher primary productivity.   The food web in these
regions often supports aggregations of higher predators such as fish,
marine mammals as well as fisheries. In addition to the different
mechanisms that result in increased primary production, there is also
the potential for bio-physical coupled aggregations of zooplankton that
serve as trophic subsidies fueling fish stocks and upper trophic levels
of top predators and marine mammals (Genin, 2004).

The bio-geographic, geologic and oceanographic settings of the study
area, accentuated by the presence of the Nazare Canyon, induce dynamic
ecological processes and are crucial for supporting high levels of
biological productivity, biodiversity and many ecosystem services. Much
of the local economy is linked to fisheries since commercially valuable
species occur here and it is important to ensure their
sustainability. Small pelagic fish are among the most important: sardine
(Sardina pilchardus), Atlantic mackerel (Scomber scombrus), chub
mackerel (Scomber japonicus) and horse mackerel (Trachurus trachurus).
The field experiment will advance on the scientific knowledge of the
dynamics of the oceanographic processes, and the bio-physical coupling
that supports the biological productivity in the study area.
Observations: adaptive sampling by robotic assets with continuous
acquisition of data on essential ocean variables (EOVs) – salinity,
temperature, oxygen, phytoplankton biomass,; complemented by ship-board
measurements (e.g. nutrients, phytoplankton functional type (using
spectrofluorescence, flow cytometry and HPLC pigments), photosynthetic
capacity, zooplankton biomass, samples for metagenomics of phyto and
zooplankton.  The shipboard measurements can be used postpriori to
validate the measurements by the autonomous platforms and model output.
Modelling : primary productivity ( maybe can be developed towards
predicting zooplankton aggregations by integrating ship-board
measurements?)  While a fully coupled physical/biological assimilation
model maybe beyond the scope of this project, \proj will allow the
exploration of the approaches to achieve that end.  Possible extensions
of the field experiment – may contribute to the definition of predictive
models helpful for ecosystems management and sea related bio-economic
activities. (e.g. development of FISHERIES MANAGEMENT ADVISOR TOOL based
on modelling approaches to assist the dynamic management of fisheries in
the Nazare Canyon-Berlengas area. This type of tools are based on (near)
real-time ocean data, species occurrence records and predictive habitat
techniques using ecological niche modelling (ENM); the aim is to
optimize the fisheries harvest of target species while minimizing
bycatch and fisheries interaction with non-target species ( e.g. marine
mammals and turtles))

