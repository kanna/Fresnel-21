% What and how will we model the oceanographic domain, what tools, methods and what kinds of predictions will the models make?

\subsection{Modeling}

\proj has two key objectives: one, to test observational strategies
and evaluate the capacity to characterize and forecast the evolution
of the prevailing coastal ocean conditions using a high resolution
limited area model with assimilation. And second, to close the
observation-prediction loop by adaptively driving the sampling of
robotic vehicles, to enhance model skill.

The expected improvement in these objectives will reinforce the
ability to support different sectors in the area -- from
security/defense and crisis response operations, to fisheries,
environmental management and general coastal and nautical
communities. It will also contribute to a significant improvement in
the our understanding of the combined physical and biogeochemical
processes that in the \naz area, contribute to drive rapid
development at all levels of the trophic chain and shape the coastal
ocean environment and associated ecosystems.

Central to this effort will be the use of the MIT Multidisciplinary
Simulation, Estimation, and Assimilation System (\msee)
\cite{haley10,mseas10}. \mse consists of a set of mathematical models
and computational methods for ocean predictions and dynamical
diagnostics, for data assimilation and data-model comparisons, and
secondarily for the optimization and adaptive sampling by autonomous
systems. It has been used for fundamental research and for realistic
simulations and predictions including monitoring \cite{lermusiaux07},
real-time acoustic-ocean predictions \cite{xu08,lermusiaux10} and
environmental management \cite{cossarini09}.

Several dynamical models are part of \msee, including a free-surface
primitive-equation model which uses implicit two-way nesting
\cite{haley10}. This multiscale free-surface code builds on the
primitive-equation model of the Harvard Ocean Prediction System
(\texttt{HOPS}) \cite{robinson96,haley99}.  Additionally, barotropic
tides are calculated from an inverse tidal model
\cite{logutov08a,logutov08b}.

A central component in \mse uncertainty prediction is the Error
Subspace Statistical Estimation for ensemble forecasting and data
assimilation developed by Lermusiaux
\cite{lermusiaux06,lermusiaux07}. \mse integrates a generalized
biogeochemical model that can be adaptive at different
levels. Depending on the number and type of state variables assigned
the generalized biological model can become the classical NPZ model,
the McGillicuddy model \cite{mcgillicuddy95}, the Anderson model
\cite{anderson00}, the Dussenberry-Lermusiaux model
\cite{becsiktepe03}, the Fasham model \cite{fasham90} or the Tian
model \cite{tian00,tian01}.

\mse was applied extensively in several scenarios and applications,
including for physical-biogeochemical forecasts for the Philippines
archipelago \cite{lermusiaux11}, in uncertainty forecasts for the
Taiwan region during the Quantifying, Predicting and Exploiting
Uncertainty (QPE) DRI \cite{gawarkiewicz11} and in stochastic
reachability, path planning and adaptive sampling forecasts for
gliders and floats during NASCar \cite{lermusiaux17}. The adaptive
sampling and path planning capacities developed in \mse will be of
importance during the observational program conducted by \proje,
providing key guidance on the operational use of different
observational assets to meet the key scientific questions defined for
the study area.

% The core of the system is a free surface primitive
% equation model \kc{Continue MODEL DESCRIPTION MSEAS} integrates a
% biogeochemical model developed initially by Dusenberry and later
% improved by Lermusiaux \cite{becsiktepe03} which integrates 5
% compartments (chlorophyll, phytoplankton based on nitrate,
% phytoplankton based on ammonia, zooplankton, nitrate, ammonia and
% detritus) and 30 parameters.

\proj will provide the opportunity to implement \mse at Instituto
Hidrogr\'{a}fico (\inste) leveraging work between IH and MIT teams
during the preparatory phase of the exercise. The model will be
initialized using data collected during the first (precursor survey --
see Table \ref{tab:tasks}) phase of the exercise combined with remote
sensing data and historical observations available for this regional
area and run operationally for the complete extent of the
experiment. Forecasts of the evolution of the physical and
biogeochemical fields will be used during the exercise to fine-tune
the observation program and in the planning of autonomous vehicles
operations.

A second and simpler model, \texttt{HOPS}, will also be used during
\proje, which is a rigid lid primitive equations model that integrates
an optimal interpolation (OI) assimilation scheme. \texttt{HOPS} has
been used at \inst for the operational support during Navy exercises,
major crises at sea or in the support of research activities. It has a
relatively modest need for computational resources which are available
onboard research vessels. \texttt{HOPS} is also equipped with a
hierarchy of biogeochemical models that comprise the majority of the
models available in \msee.

This effort will be guided by the use of the previous implementation
of \texttt{HOPS} for the \naz area. The model was configured using a
high resolution Cartesian horizontal grid (300m resolution), 30 double
sigma levels in the vertical and a realistic topography of the \naz
canyon area conditioned to meet the hydrostatic consistency
requirements.  Atmospheric forcing conditions will be provided by
available operational forecast models. Operational models at \inst use
the analysis and forecasts to 6 days provided by the European Centre
for Medium Range Weather Forecast (ECMWF) and made available by the
Portuguese node at Instituto Portugu\^{e}s do Mar e da Atmosfera
(\texttt{IPMA} -- the Portuguese NOAA).  Boundary conditions at the
open boundaries of the models will be provided by available regional
models. These include the results provided by the Copernicus Marine
Service for the IBIROSS area which is a NEMO model (analysis and
forecasts to 5 days, 3km resolution and 50 vertical levels) or the
results from the HYCOM model configuration run operationally at \inst
covering the Iberian area and Moroccan margin.  Boundary conditions at
the coastal boundary associated with freshwater outflows due to some
of the small rivers that directly impact the model domain area will be
incorporated using the information available about discharge
conditions provided by the Sistema Nacional de Informa\c{c}\~{a}o de
Recursos Hidricos\footnote{\url{https://snirh.apambiente.pt/}}
available from the Portuguese Environment Agency.

% \kc{Data sets to model initialization and assimilation \ldots}

