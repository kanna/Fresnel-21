What and how will we model the oceanographic domain, what tools, methods and what kinds of predictions will the models make?

As stated in the introduction a central objective of \proj is to test
observation strategies and evaluate the capacity to characterize and
forecast the evolution of the prevailing coastal ocean conditions using
a high resolution limited area model with assimilation. The expected
improvement of this capacity will reinforce the ability to support
different sectors in the area (from defense and crisis response
operations to support to fisheries, environmental managers and general
coastal and nautical communities). It will also contribute to a
significant improvement in the present understanding of the combined
physical and biogeochemical processes that, in this area, contribute to
drive rapid development at all levels of the trophic chain / shape the
coastal ocean environment and associated ecosystems.  Central to this
effort is the use of the MIT Multidisciplinary Simulation, Estimation,
and Assimilation System – MSEAS (MSEAS Group, 2010). The core of the
system is a free surface primitive equation model Continue MODEL
DESCRIPTION MSEAS integrates a biogeochemical model developed initially
by Dusenberry and latter improved by Lermusiaux which integrates 5
compartments (chlorophyll, phytoplankton based on nitrate, phytoplankton
based on ammonia, zooplankton, nitrate, ammonia and detritus) and 30
parameters.

\proj will provide the opportunity to implement MSEAS at Instituto
Hidrografico profiting for the articulation between IH and MIT teams
during the preparatory phase of the exercise and the

A second and simpler model will also be used during \proje.  The Harvard
Ocean Prediction System (HOPS) is a rigid lid primitive equations model
that integrates an OI assimilation scheme. HOPS have been used in IH for
the operational support during navy exercises, major crises at sea or in
the support of research activities. It can be used in a very flexible
way for example using relatively modest computational resources existent
onboard research vessels. HOPS is also equipped with the initial version
of the Dusenberry biogeochemical model.

The implementation of both these models will be guided by the experience
gathered by a previous implementation of the HOPS model for this same
area. The model was configured using a high resolution Cartesian
horizontal grid (300m resolution), 30 double sigma levels in the
vertical and a realistic topography of the Nazare Canyon area
conditioned to meet the hydrostatic consistency requisites.  Atmospheric
forcing conditions will be provided by available operational forecast
models. Operational models at Instituto Hidrográfico use the analysis
and forecasts to 6 days provided by the European Centre for Medium Range
Weather Forecast (ECMWF) and made available by the portuguese node at
(Instituto Português do Mar e da Atmosfera).  Boundary conditions at the
open boundaries of the models will be provided by available regional
models. These include the results provided by the Copernicus Marine
Service for the IBIROSS area which are NEMO model (analysis and
forecasts to 5 days, 3km resolution and 50 vertical levels) or the
results from the HYCOM model configuration run operationally at
Instituto Hidrográfico covering the Iberian area and Moroccan margin.
Boundary conditions at the coastal boundary associated with freshwater
outflows due to some of he small rivers that directly affect the model
domain area will be incorporated using the information available about
discharge conditions provided by the National Information System on
Hydric Resources portal made available by the Portuguese Agency for the
Environment.

Data sets to model initialization and assimilation

\begin{table}[!t]
  \centering
  \vspace{-0.5cm}
  \begin{tabular}{|p{4cm}|p{4cm}|p{4cm}|p{4cm}|}\hline 
    % \rowcolor{Gray}
    \bfseries  &\bfseries Week 1 &\bfseries Week 2 &\bfseries Week 3 \\
    \hline
    Main Goals& Precursor survey; 
                To collect data that allow to build initialization
                fields and define model parameters. To calibrate sensor response (phytoplankton from
                fluorometry, zooplankton from VMADCP)& Update Survey 1:
                                                       AUV operations; potential additional measurements using small boats
                                                       and low cost measurements & Update Survey 2:
                                                                                   Ship measurements during dedicated survey, AUV operations from ship and small boats\\
    \hline
    Observations at sea&&&\\
    \hline
    AUVs&&&\\
    \hline
    Gliders&&&\\
    \hline
    UAVs&&&\\
    \hline
    ASVs&&&\\
    \hline
    Ship-based observations& Ship crossing the area, deploy Multipametric buoy M2
                             CTDs/Water Sampling at key positions
                             VMADCP collected in area/key points&& CTD/LADCP profiles
                                                                   Rosette
                                                                   Samples
                                                                   for
                                                                   nutrients/phyto/zoo-plankton
                                                                   VMADCP on transit and in station. 
                                                                   Daily transmission of  CTD casts and VMADCP data to \inst\\
    \hline
    Small boat operations&CTD + fluorometry + light profiles at key
                           stations. Water samples at key points for
                           nutrients/phyto+zoo-plankton Vertical net
                           samples for phyto+zoo plankton&&\\
    \hline    
    Laboratory Analysis&Samples analyzed for Nutrients at \inst Labs.
                         Samples analyzed for phyto/zoo-plankton&&Post cruise:
                                                                   Samples
                                                                   analyzed
                                                                   for
                                                                   Nutrients
                                                                   at
                                                                   \inst
                                                                   Labs
                                                                   Samples
                                                                   analyzed
                                                                   for
                                                                   phyto/zoo-plankton\\ 
    \hline
    Remote Sensing&SST, Chl, Turbidity images (Sentinel) used to
                    identify the main features and select key points for
                    observation. 
                    Surface fields combined with observations in key
                    points to build initial 3D fields to be used in the
                    models.&SST, Chl, altimetry data used in assimilation
                             Turbidity images used to track impacts (if
                             any) of local rivers and guide
                             observations.&SST, Chl, altimetry data used in assimilation
                                            Turbidity images used to
                                            track impacts (if any) of
                                            local rivers and guide
                                            observations.\\
    \hline
    Models&Build initialization and initial assimilation fields. 
            Start model runs (end of the week).&&\\
    \hline
  \end{tabular}
  \caption{Tasks and activities in \proje.}
  \label{tab:tasks}
\end{table}
