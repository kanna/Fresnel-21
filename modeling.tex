% What and how will we model the oceanographic domain, what tools, methods and what kinds of predictions will the models make?

\subsection{Modeling}

There are two key objectives of \proje: one, to test observational
strategies and evaluate the capacity to characterize and forecast the
evolution of the prevailing coastal ocean conditions using a high
resolution limited area model with assimilation. And second, to close
the observation-prediction loop by adaptively driving the sampling of
robotic vehicles, to enhance model skill.

The expected improvement in these objectives will reinforce the
ability to support different sectors in the area -- from defense and
crisis response operations, to fisheries, environmental managers and
general coastal and nautical communities. It will also contribute to a
significant improvement in the present understanding of the combined
physical and biogeochemical processes that, in the \naz area,
contribute to drive rapid development at all levels of the trophic
chain and shape the coastal ocean environment and associated
ecosystems.

Central to this effort will be the use of the MIT Multidisciplinary
Simulation, Estimation, and Assimilation System (MSEAS)
\cite{haley10}. The core of the system is a free surface primitive
equation model \kc{Continue MODEL DESCRIPTION MSEAS} integrates a
biogeochemical model developed initially by Dusenberry and later
improved by Lermusiaux \cite{becsiktepe03} which integrates 5
compartments (chlorophyll, phytoplankton based on nitrate,
phytoplankton based on ammonia, zooplankton, nitrate, ammonia and
detritus) and 30 parameters. \proj will provide the opportunity to
implement MSEAS at Instituto Hidrogr\'{a}fico (\inste) leveraging work
between IH and MIT teams during the preparatory phase of the exercise.

A second and simpler model will also be used during \proje.  The
Harvard Ocean Prediction System (HOPS) \cite{robinson96} is a rigid
lid primitive equations model that integrates an optimal interpolation
(OI) assimilation scheme. HOPS has been used at \inst for the
operational support during Navy exercises, major crises at sea or in
the support of research activities. It has a relatively modest need
for computational resources which are available onboard research
vessels. HOPS is also equipped with the initial version of the
Dusenberry biogeochemical model.

This effort will be guided by the use of the previous implementation
of HOPS for the \naz area. The model was configured using a high
resolution Cartesian horizontal grid (300 m resolution), 30 double
sigma levels in the vertical and a realistic topography of the \naz
Canyon area conditioned to meet the hydrostatic consistency
requirements.  Atmospheric forcing conditions will be provided by
available operational forecast models. Operational models at \inst use
the analysis and forecasts to 6 days provided by the European Centre
for Medium Range Weather Forecast (ECMWF) and made available by the
Portuguese node at Instituto Portugu\^{e}s do Mar e da Atmosfera (IPMA
-- the Portuguese NOAA).  Boundary conditions at the open boundaries
of the models will be provided by available regional models. These
include the results provided by the Copernicus Marine Service for the
IBIROSS area which are a NEMO model (analysis and forecasts to 5 days,
3km resolution and 50 vertical levels) or the results from the HYCOM
model configuration run operationally at \inst covering the Iberian
area and Moroccan margin.  Boundary conditions at the coastal boundary
associated with freshwater outflows due to some of he small rivers
that directly affect the model domain area will be incorporated using
the information available about discharge conditions provided by the
National Information System on Hydric Resources portal made available
by the Portuguese Agency for the Environment.

% \kc{Data sets to model initialization and assimilation \ldots}

