\section*{Project Abstract}
\label{sec:abs}
\addcontentsline{toc}{section}{Project Abstract}

The coastal ocean shapes the two-way interaction between the deep
ocean/ocean basins and the coastal populations and human societies. They
determine how anthropogenic influences originating from the continents
are redistributed, while impacting the maritime environment. Coastal
ocean processes directly impact and influence how humans interact with
the oceans, whether for civilian maritime needs such as fishing,
recreation or extraction of minerals, or for security and dual-use needs
related to monitoring and surveillance. It is critical for us to
understand and ultimately predict the evolution of the different
processes in this dynamic environment. Yet our predictability and
consequent understanding of this complex environment has been lagging in
part because sufficient inter-disciplinary studies across biology and
physics have been lacking, in part because of tools and methods have not
been fully brought to bear on arguably a difficult domain to work in.

\proj proposes to to close the observe-assimilate-predict-sample loop
by demonstrating the applicability of adaptively controlled marine
robots in the aerial, surface and underwater domains, while sampling
the upper water-column \emph{'at the right place and time'} driven by
ocean models with increasing predictive skill. In doing so, we wish to
increase predictive skill of ocean models, leverage advances in
Artificial Intelligence and decision-making, robotics and bring to
bear recent advances in Machine Learning for adaptation and
prediction. \proj involves a diverse group of seasoned researchers
working across traditional disciplinary boundaries. The tight
integration between model prediction and assimilation that we foresee
occurring as part of this effort, will be enhanced so as to provide
realistic forecasts of a range of biophysical variables including
temperature, salinity, wind, surface and subsurface currents and
bio-optical properties. These in turn will be used to intelligently
target sampling with these multi-domain platforms in the air, ocean
surface and underwater, augmented by satellite remote sensing
including from a recently launched multi-spectral sensor on a Small
Satellite.

The novelty of this proposed effort is in the integrative aspects of a
field exercise which will allow \proj to leap-frog experimental
design, autonomous operations, assimilation, modeling and prediction
in ways not done before. The project will outreach substantially with
local authorities, subsistence fishermen and an NGO in the domain of
operation in \naze, Portugal and engage local middle and high-school
students, along the lines of previous such field experiments.
