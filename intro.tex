\section{Introduction}

% Discuss:

% the problem domain, 
% The scientific  hypothesis for FRESNEL (biology, physics, technology), 
% the technology and 
% the need for the experiment. 



In \proj (Field expeRiments for modEling, aSsimilatioN and adaptivE
sampLing) we propose to close the observe-assimilate-predict-sample loop
in the coastal ocean. Model skill, especially in the coastal ocean, is
highly dependent on sparsely sampled in-situ measurements and remote
sensing data (when available). Typically satellite data augmented by
some fixed assets such as buoys, or Lagrangian floats constitute the
bulk of the observations with a few ship-board measurements providing
validation of the model. However few models provide assimilated
information at small spatial scales ($sim 100$'s of meters) relevant to
short term tactical planning.

Our objective in \proj is to demonstrate the applicability of adaptively
controlled marine robots in the aerial, surface and underwater domains,
while sampling the upper water-column 'at the right place and time'
driven by ocean models with increasing predictive skill. Fig. 1 shows a
conceptual view of the proposed field experiment. In doing so, we wish
to:

\begin{enumerate}

\item Increase the predictive skill of coastal ocean models

\item Leverage the advances in Artificial Intelligence driven
  dynamic-decision making

\item Close the sense-assimilate-predict-sample loop
  
\item Advance the efforts to bring modern Machine Learning methods for
  adaptation and prediction in the advancement of our understanding of
  coastal ocean processes
  
\end{enumerate}


\begin{figure}[!b]
  \centering
  \includegraphics[scale=0.15]{fig/ensemble.jpg}
  \caption{\proj will integrate ocean models with adaptive robotic vehicles
    in the coastal ocean, to increase model skill while increasing
    model accuracy and prediction with a tight loop.}
  \label{fig:block-diag}
\end{figure}


The \proj team is interdisciplinary and includes members from diverse
backgrounds in biological and physical oceanography including modeling,
Artificial Intelligence decision making, Control Theory and Operational
Oceanography. The members of the team have known each other for some
time and in many instances have worked together for more than a decade.
The proposed effort is integrative, in that work in these diverse fields
has advanced in siloed environments or other domains; with \proj we plan
to bring together the disparate elements in ways hitherto not done
before, especially the tight integration of AI-driven marine robotic
vehicles in the aerial, surface and underwater domains with modeling and
control. In doing so, the tight integration between modelprediction and
assimilation will be enhanced so as to provide realistic forecasts of a
range of biophysical variables including temperature, salinity, wind,
surface and subsurface currents and bio-optical properties. These in
turn will be used to intelligently target sampling with these
multi-domain platforms. Important outcomes of this proposed project
include:

\begin{itemize}

\item rapid assessment of environmental state using state of the art
  methods in modeling, control and sampling with minimal human
  intervention
  
\item increasing model prediction skill with targeted sampling to reduce
  uncertainty
  
\item real-time decision support to determine appropriate mix of robotic
  or manned assets (e.g. small boats or research vessel) for targeted
  sampling
  
\item demonstration of coordinated observations with an ensemble of
  robotic vehicles with minimal human intervention

\end{itemize}  

The novelty of this proposed effort is in the integrative aspects of
this field exercise; while individual aspects might have been
demonstrated in other field experiments (including by members of this
team), \proj will leap-frog experimental design, autonomous
operations, assimilation, modeling and prediction in ways not done
before.

