\section{Management Approach}

This collaborative effort is in part to advance our understanding of
the bio-geophysical processes in the \naz area, while advancing the
observational technology and capacity in the coastal ocean. 

\proj will be managed by the PI (Rajan, a US national) working closely
with the collaborators from the \texttt{Univ. of Porto},
\texttt{Instituto Hidrogr\'{a}fico}, \texttt{Univ. of Aveiro},
\texttt{Columbia Univ.}, \texttt{MIT} and \texttt{SOCIB}.  All key
personnel have known and worked with one another, most over a decade,
some at sea. \univ will be the prime, where the PI has a Visiting
Professor position. The roles and responsibilities of the key
personnel are shown in Table \ref{tab:roles}.

\begin{table}[!t]
  \centering
  % \vspace{-0.5cm}
  \footnotesize{
  \begin{tabular}{|p{2.7cm}|p{2.5cm}|p{5cm}|p{4.5cm}|}\hline 
    \rowcolor{Gray}
    \bfseries Name& \bfseries Institution&\bfseries Expertise/Qualification &\bfseries Contributions\\
    \hline
    Kanna Rajan&\orge, US and \texttt{Univ. of Porto}, Portugal&PI, Autonomy, Adaptive Sampling&Organization, reporting, experiment design, outreach\\
    \hline
    Jo\~ao Sousa&\texttt{Univ. of Porto}, Portugal&Co-PI, Operations Lead, CONOPS (concept of operations) design and
            implementation, software eng.
                                    &Organization, Aerial/surface/underwater
                                      vehicles, comms\\
    \hline
    Jo\~ao Vitorino&\texttt{Instituto Hidrogr\'{a}fico}, Portugal&Physical Oceanography, Modeling&Observation assimilation, prediction,
                                                            local outreach\\
    \hline
    Marina Cunha&\texttt{Univ. of Aveiro}, Portugal&Coastal Ecology&Biological sampling, lab analysis\\
    \hline
    Joaqu\'{i}n Tintor\'{e}&\texttt{SOCIB}, Spain &Physical Oceanography &Experiment
                                                          design,Gliders and operations\\
    \hline
    Ajit Subramaniam&\texttt{Columbia Univ.}, US&Biological Oceanography&CONOPS, sampling
                             algorithms\\
    \hline
    Pierre Lermusiaux&\texttt{MIT}, US&Modeling, Machine Learning and entropy
                             reduction&Modeling support\\
    \hline
  \end{tabular}
  \caption{Roles and responsibilities and in-kind contributions in
    \proj for the proposed 2021 Sept-Oct field experiment. All team
    members will be collaboratively involved in experiment design and
    post experiment publishing.}
  \label{tab:roles}
}
\end{table}

\paragraph{Roles and responsibilities} \univ will be the lead
organization, provide aerial, surface and underwater vehicles,
communication equipment and command/control software while
coordinating all activities. Funding requested will primarily support
students and staff in Porto for software development and operations.
\inst will provide the assimilation, modeling and prediction with
shore-side models. In addition \inst will conduct CTD and vessel
mounted observations onboard the research vessel and also provide
access to their research vessel as well as a RHIB or rigid boat. \mit
will support \inst for \texttt{HOPS} modeling and augmentation for ML
capabilities. \colo and \ave will work on making bio-optical and
biological measurements and providing analysis and data to augment
\inst modeling. Table \ref{tab:roles} summarizes the roles and
contributions of all partners.


\subsection{Current Pending Project and Proposal Submissions}

No proposals related to \proj have been submitted to any US government
agency. However, this proposal is in part to demonstrate a portion of
a much larger concept in the form of the \met project\footnote{See US
  National Academies 'Ocean Shot' presentation at
  \url{https://vimeo.com/510346249}} which is being proposed to
private philanthropies for funding. \met aims to provide a new
paradigm of portable coastal ocean observation using space, aerial,
surface and underwater vehicles, coupled to assimilative models.

\subsection{Relevant Experience}

% A description of the Offeror's Government contracts (both prime and
% major subcontracts) received during the past three (3) years, which
% are similar in nature to the effort being proposed.  Include contract
% number and the name, phone number and email address of the technical
% point of contact.  If necessary, add additional rows.

The PI and co-PI do not have any US govt. contracts or had any
contracts during the past 3 years. 