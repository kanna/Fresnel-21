\section{Management Approach}

This collaborative effort is in part to advance our understanding of
the bio-geophysical processes in the \naz area, while advancing the
observational technology and capacity in the coastal ocean.  \proj
will be managed by the PI (Rajan, a US national) working closely with
collaborators from the \texttt{Univ.~of Porto}, \texttt{Instituto
  Hidrogr\'{a}fico}, \texttt{Univ.~of Aveiro}, \texttt{Columbia
  Univ.}, \texttt{MIT} and \texttt{SOCIB}.  All key personnel have
known and worked with one another, most over a decade, some at
sea. \univ will be the prime, where the PI is a Visiting faculty.

\begin{table}[!t]
  \centering
  % \vspace{-0.5cm}
  \footnotesize{
  \begin{tabular}{|p{2.7cm}|p{2.5cm}|p{5cm}|p{4.5cm}|}\hline 
    \rowcolor{Gray}
    \bfseries Name& \bfseries Institution&\bfseries Expertise/Qualification &\bfseries Contributions\\
    \hline
    Kanna Rajan&\texttt{Univ. of Porto}, Portugal&PI, Autonomy,
                                                   Adaptive
                                                   Sampling&Organization,
                                                             reporting,
                                                             experiment
                                                             design,
                                                             outreach\\ 
    \hline
    Jo\~ao Sousa&\texttt{Univ. of Porto}, Portugal&Co-PI, Operations
                                                    Lead, CONOPS
                                                    (concept of
                                                    operations) design
                                                    and 
            implementation, software eng.
                                    &Organization, Aerial/surface/underwater
                                      vehicles, comms\\
    \hline
    Jo\~ao Vitorino&\texttt{Instituto Hidrogr\'{a}fico},
                     Portugal&Physical Oceanography,
                               Modeling&Observation assimilation and modeling,
                                         CONOPS, prediction, local outreach\\
    \hline
    Marina Cunha&\texttt{Univ. of Aveiro}, Portugal&Coastal Ecology&Biological sampling, lab analysis\\
    \hline
    Joaqu\'{i}n Tintor\'{e}&\texttt{SOCIB}, Spain &Physical Oceanography &Experiment
                                                          design,Gliders and operations\\
    \hline
    Ajit Subramaniam&\texttt{Columbia Univ.}, US&Biological Oceanography&CONOPS, sampling
                             algorithms\\
    \hline
    Pierre Lermusiaux&\texttt{MIT}, US&Modeling, Machine Learning and entropy
                             reduction&Modeling support\\
    \hline
  \end{tabular}
  \caption{Roles and responsibilities and in-kind contributions in
    \proj for the proposed 2023 March field experiment. All team
    members will be collaboratively involved in experiment design and
    post experiment publishing.}
  \label{tab:roles}
}
\end{table}

\paragraph{Roles and responsibilities} \univ will be the lead
organization, provide aerial, surface and underwater vehicles,
communication equipment and command/control software while
coordinating all activities. Funding requested will primarily support
students and staff in Porto for the expected period of the experiment
for $\sim 20$ days. Pre-experiment software development will be done
at \univ leveraging existing funding from other sources. \inst will
provide the assimilation, modeling and prediction with shore-side
models working in conjunction with \mite. In addition \inst will
conduct CTD and vessel mounted observations onboard the research
vessel; in addition to access to the research vessel for the rest of
the team, \inst will also make provision for a RHIB or rigid
boat. \mit will support \inst for \mse and \texttt{HOPS} modeling and
augmentation for ML capabilities. \colo and \ave will work on making
bio-optical and biological measurements and providing analysis and
data to augment \inst modeling. \soc will provide a glider and
personnel for preparation, deployment, operations and support of the
vehicle as well as helping with the analysis of glider data during and
after the experiment. Table \ref{tab:roles} summarizes the roles,
contributions and responsibilities of key personnel.

\paragraph{Budget Request} Our funding request is modest and highly
leveraged with equipment and people who are keen on participating in
this experiment. No salaries, or expensive ship-times are being
requested, instead we will be using existing assets which will be in
the \naz area for fixed periods of time and use robotic hardware and
software from \univ and \soce. Consequently, our budget reflects costs
for consumables, insurance, SatComs, and travel only. \univ is the
primary beneficiary of this request with minor requests for \inste,
\avee, \colo and \soc (Table \ref{tab:budget}). No funding is being
requested for \mite.

\begin{table}[!t]
  \centering
  % \vspace{-0.5cm}
  \footnotesize{
  % \begin{tabular}{|p{3.3cm}|p{1.3cm}|p{1.4cm}|p{8cm}|}
  \begin{tabular}{|p{3.3cm}|p{1.3cm}|r|p{8cm}|}
    % \multicolumn{1}{l}{r}{l}
    \hline 
    \rowcolor{Gray}
    \bfseries Item& \bfseries Inst.&\bfseries Cost &\bfseries Comment\\
    \hline
    Small-boat rental&\univ&\$16,958&19-day rental\\
    \hline
    Semi-rigid small-boat&\inst&\$1,785&\inst small-boat for ops with
                                         research vessel\\
    \hline    
    Insurance (AUVs)&\univ&\$31,654&Insurance costs for 7 AUVs\\
    \hline
    Insurance (UAVs)&\univ&\$1,190&Insurance costs for 2 UAVs\\
    \hline
    Lodging&\univ&\$7,616&Costs for lodging in \naz area 8 personnel for 20 days\\
    \hline
    Subsistence&\univ&\$11,424&Costs for subsistence in \naz area for 8
                                personnel for 20 days\\
    \hline
    Communication costs&\univ&\$11,870&SatComs for AUVs for 19
                                        operational days\\
    \hline
    Rental Cars&\univ&\$7,140&Rental cars/vans for transporting
                               equipment from Porto\\
    \hline
    Gas \& Tolls&\univ&\$1,179&To support travel with rentals from Porto\\
    \hline
    Consumables&\univ&\$10,710&Short lifetime needs for field operations
                                with autonomous platforms such as
                                batteries, fins, servo-motors for the
                                rudders, bolts, o-rings, etc\\ 
    \hline    
    Cellular Comms&\univ&\$893&\\
    \hline    
    Sample Processing&\ave&\$5,950&Biological water sample processing
                                    during and post-experiment\\
    \hline    
    Sample Processing&\inst&\$1,190&Consumables for BGC water sample
                                     analysis at \inst\\
    \hline    
    Glider batteries&\soc&\$3,570&Consumables for glider ops\\
    \hline
    Glider comms&\soc&\$4,760&Iridium SatCom costs for glider ops\\
    \hline    
    Travel from US&\colo \& NYC&\$4,046&Travel from US and lodging for
                                          Subramaniam \& Rajan (based
                                         in NYC)\\
    \hline
    \multicolumn{1}{|r|}{\textbf{Total}}&&\$121,933&\\
    \hline    
  \end{tabular}
  \caption{Budget details for \proj field experiment in USD.}
  \label{tab:budget}
}
\end{table}

